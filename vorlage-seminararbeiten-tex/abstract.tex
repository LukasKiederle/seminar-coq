\section*{Kurzfassung}
\thispagestyle{empty}


Aus Zeit- und Kostengründen beim Entwickeln und Testen von immer komplexer werdenden Systemen, werden Tools zur Programmcodeverifikation immer relevanter.
Diese Tools ermöglichen das Schreiben von Programmen, welche mathematisch und maschinell geprüft sind. Dadurch ist sichergestellt, dass das beschriebene Programm sich auch wie gewünscht, verhält.\\
Ziel dieser Arbeit ist es, einen sowohl theoretischen als auch technischen Einblick in die Programmcodeverifikation mit dem Proof Assistent Tool Coq darzustellen.
Als Einstieg werden die grundlegenden Begriffe geklärt und ein kurzer Überblick über Tools in diesem Fachbereich dargestellt. Dabei wird insbesondere auf Coq eingegangen.\\
Um ein Verständnis zu bekommen, wie ein Proof Assistent Tool die Qualität von Programmcode sicherstellt, müssen zunächst die Grundlagen dieser Sprache anhand von Beispielen erklärt werden. Anschließend wird näher auf das Zusammenspielen zwischen Programmcode und Proof Assistent eingegangen.\\
Es gibt bereits einige sehr erfolgreiche Forschungsprojekte, die Coq im Einsatz haben. Diese werden abschließend vorgestellt. Schlussendlich wird ein Fazit inklusive Ausblicks in Hinsicht auf die Verwendbarkeit von Proof Assistent Tools gezogen.
\bigskip

\noindent
Schlagworte:
\begin{itemize}
	\item Proof Assistent
	\item Coq
	\item Programcodeverifikation
	\item Formale Verifikation
\end{itemize}